% Options for packages loaded elsewhere
\PassOptionsToPackage{unicode}{hyperref}
\PassOptionsToPackage{hyphens}{url}
%
\documentclass[
]{article}
\usepackage{amsmath,amssymb}
\usepackage{iftex}
\ifPDFTeX
  \usepackage[T1]{fontenc}
  \usepackage[utf8]{inputenc}
  \usepackage{textcomp} % provide euro and other symbols
\else % if luatex or xetex
  \usepackage{unicode-math} % this also loads fontspec
  \defaultfontfeatures{Scale=MatchLowercase}
  \defaultfontfeatures[\rmfamily]{Ligatures=TeX,Scale=1}
\fi
\usepackage{lmodern}
\ifPDFTeX\else
  % xetex/luatex font selection
\fi
% Use upquote if available, for straight quotes in verbatim environments
\IfFileExists{upquote.sty}{\usepackage{upquote}}{}
\IfFileExists{microtype.sty}{% use microtype if available
  \usepackage[]{microtype}
  \UseMicrotypeSet[protrusion]{basicmath} % disable protrusion for tt fonts
}{}
\makeatletter
\@ifundefined{KOMAClassName}{% if non-KOMA class
  \IfFileExists{parskip.sty}{%
    \usepackage{parskip}
  }{% else
    \setlength{\parindent}{0pt}
    \setlength{\parskip}{6pt plus 2pt minus 1pt}}
}{% if KOMA class
  \KOMAoptions{parskip=half}}
\makeatother
\usepackage{xcolor}
\usepackage[margin=1in]{geometry}
\usepackage{color}
\usepackage{fancyvrb}
\newcommand{\VerbBar}{|}
\newcommand{\VERB}{\Verb[commandchars=\\\{\}]}
\DefineVerbatimEnvironment{Highlighting}{Verbatim}{commandchars=\\\{\}}
% Add ',fontsize=\small' for more characters per line
\usepackage{framed}
\definecolor{shadecolor}{RGB}{248,248,248}
\newenvironment{Shaded}{\begin{snugshade}}{\end{snugshade}}
\newcommand{\AlertTok}[1]{\textcolor[rgb]{0.94,0.16,0.16}{#1}}
\newcommand{\AnnotationTok}[1]{\textcolor[rgb]{0.56,0.35,0.01}{\textbf{\textit{#1}}}}
\newcommand{\AttributeTok}[1]{\textcolor[rgb]{0.13,0.29,0.53}{#1}}
\newcommand{\BaseNTok}[1]{\textcolor[rgb]{0.00,0.00,0.81}{#1}}
\newcommand{\BuiltInTok}[1]{#1}
\newcommand{\CharTok}[1]{\textcolor[rgb]{0.31,0.60,0.02}{#1}}
\newcommand{\CommentTok}[1]{\textcolor[rgb]{0.56,0.35,0.01}{\textit{#1}}}
\newcommand{\CommentVarTok}[1]{\textcolor[rgb]{0.56,0.35,0.01}{\textbf{\textit{#1}}}}
\newcommand{\ConstantTok}[1]{\textcolor[rgb]{0.56,0.35,0.01}{#1}}
\newcommand{\ControlFlowTok}[1]{\textcolor[rgb]{0.13,0.29,0.53}{\textbf{#1}}}
\newcommand{\DataTypeTok}[1]{\textcolor[rgb]{0.13,0.29,0.53}{#1}}
\newcommand{\DecValTok}[1]{\textcolor[rgb]{0.00,0.00,0.81}{#1}}
\newcommand{\DocumentationTok}[1]{\textcolor[rgb]{0.56,0.35,0.01}{\textbf{\textit{#1}}}}
\newcommand{\ErrorTok}[1]{\textcolor[rgb]{0.64,0.00,0.00}{\textbf{#1}}}
\newcommand{\ExtensionTok}[1]{#1}
\newcommand{\FloatTok}[1]{\textcolor[rgb]{0.00,0.00,0.81}{#1}}
\newcommand{\FunctionTok}[1]{\textcolor[rgb]{0.13,0.29,0.53}{\textbf{#1}}}
\newcommand{\ImportTok}[1]{#1}
\newcommand{\InformationTok}[1]{\textcolor[rgb]{0.56,0.35,0.01}{\textbf{\textit{#1}}}}
\newcommand{\KeywordTok}[1]{\textcolor[rgb]{0.13,0.29,0.53}{\textbf{#1}}}
\newcommand{\NormalTok}[1]{#1}
\newcommand{\OperatorTok}[1]{\textcolor[rgb]{0.81,0.36,0.00}{\textbf{#1}}}
\newcommand{\OtherTok}[1]{\textcolor[rgb]{0.56,0.35,0.01}{#1}}
\newcommand{\PreprocessorTok}[1]{\textcolor[rgb]{0.56,0.35,0.01}{\textit{#1}}}
\newcommand{\RegionMarkerTok}[1]{#1}
\newcommand{\SpecialCharTok}[1]{\textcolor[rgb]{0.81,0.36,0.00}{\textbf{#1}}}
\newcommand{\SpecialStringTok}[1]{\textcolor[rgb]{0.31,0.60,0.02}{#1}}
\newcommand{\StringTok}[1]{\textcolor[rgb]{0.31,0.60,0.02}{#1}}
\newcommand{\VariableTok}[1]{\textcolor[rgb]{0.00,0.00,0.00}{#1}}
\newcommand{\VerbatimStringTok}[1]{\textcolor[rgb]{0.31,0.60,0.02}{#1}}
\newcommand{\WarningTok}[1]{\textcolor[rgb]{0.56,0.35,0.01}{\textbf{\textit{#1}}}}
\usepackage{graphicx}
\makeatletter
\newsavebox\pandoc@box
\newcommand*\pandocbounded[1]{% scales image to fit in text height/width
  \sbox\pandoc@box{#1}%
  \Gscale@div\@tempa{\textheight}{\dimexpr\ht\pandoc@box+\dp\pandoc@box\relax}%
  \Gscale@div\@tempb{\linewidth}{\wd\pandoc@box}%
  \ifdim\@tempb\p@<\@tempa\p@\let\@tempa\@tempb\fi% select the smaller of both
  \ifdim\@tempa\p@<\p@\scalebox{\@tempa}{\usebox\pandoc@box}%
  \else\usebox{\pandoc@box}%
  \fi%
}
% Set default figure placement to htbp
\def\fps@figure{htbp}
\makeatother
\setlength{\emergencystretch}{3em} % prevent overfull lines
\providecommand{\tightlist}{%
  \setlength{\itemsep}{0pt}\setlength{\parskip}{0pt}}
\setcounter{secnumdepth}{-\maxdimen} % remove section numbering
\usepackage{bookmark}
\IfFileExists{xurl.sty}{\usepackage{xurl}}{} % add URL line breaks if available
\urlstyle{same}
\hypersetup{
  pdftitle={Homework 2},
  pdfauthor={Mark Wilber},
  hidelinks,
  pdfcreator={LaTeX via pandoc}}

\title{Homework 2}
\author{Mark Wilber}
\date{2024-09-19}

\begin{document}
\maketitle

\section{Fitting linear models to understand within-host pathogen
growth}\label{fitting-linear-models-to-understand-within-host-pathogen-growth}

The amphibian chytrid fungus \emph{Batrachochytrium dendrobatidis} (Bd)
has led to the declines and extinctions of hundreds of amphibians across
the globe. The severity of the disease chytridiomycosis (caused by the
pathogen Bd) depends on how infected individual amphibians are. Thus, it
is important to understand the dynamics of within-host pathogen growth.

For this homework, you will analyze data on the dynamics of Bd growth on
the endangered Mountain yellow-legged frog. In this experiment, frogs
were kept at 12 C or 20 C, exposed to Bd, and swabbed every three days.
Swabbing the frog allows us to measure how much Bd is on frogs. You are
asking two questions:

\begin{enumerate}
\def\labelenumi{\arabic{enumi}.}
\tightlist
\item
  Is the growth rate of Bd on frogs different at 12 and 20 C?
\item
  Is the predicted equilibrium Bd load on frogs different at 12 C and 20
  C?
\end{enumerate}

Because disease-induced mortality is load-dependent, you ultimately what
to understand how manipulating temperature might affect infection
dynamics and mortality.

\subsection{The data}\label{the-data}

The data are given in \texttt{bd\_growth\_data.csv}. The columns are

\begin{enumerate}
\def\labelenumi{\arabic{enumi}.}
\tightlist
\item
  \texttt{logload\_t}: Natural log of Bd load on a frog at time \(t\)
\item
  \texttt{logload\_tplus1}: Natural log of Bd load on a frog at time
  \(t + 1\)
\item
  \texttt{temp}: Temperature of 12 or 20 C
\item
  \texttt{individual}: Individual ID of frogs
\end{enumerate}

\subsubsection{Question 1}\label{question-1}

Plot the relationship between \texttt{loadload\_t} and
\texttt{logload\_tplus1} for 12 and 20 C. Describe at least 2
characteristics of the relationship you are seeing.

\begin{Shaded}
\begin{Highlighting}[]
\FunctionTok{library}\NormalTok{(dplyr)}
\end{Highlighting}
\end{Shaded}

\begin{verbatim}
## 
## Attaching package: 'dplyr'
\end{verbatim}

\begin{verbatim}
## The following objects are masked from 'package:stats':
## 
##     filter, lag
\end{verbatim}

\begin{verbatim}
## The following objects are masked from 'package:base':
## 
##     intersect, setdiff, setequal, union
\end{verbatim}

\begin{Shaded}
\begin{Highlighting}[]
\NormalTok{bd\_data }\OtherTok{=} \FunctionTok{read.csv}\NormalTok{(}\StringTok{"bd\_growth\_data.csv"}\NormalTok{)}
\CommentTok{\#head(bd\_data)}

\NormalTok{bd\_12 }\OtherTok{\textless{}{-}}\NormalTok{ bd\_data }\SpecialCharTok{\%\textgreater{}\%} \FunctionTok{filter}\NormalTok{(temp }\SpecialCharTok{==} \StringTok{"12"}\NormalTok{)}
\CommentTok{\#bd\_12}

\NormalTok{bd\_20 }\OtherTok{\textless{}{-}}\NormalTok{ bd\_data }\SpecialCharTok{\%\textgreater{}\%} \FunctionTok{filter}\NormalTok{(temp }\SpecialCharTok{==} \StringTok{"20"}\NormalTok{)}
\CommentTok{\#bd\_20}
\end{Highlighting}
\end{Shaded}

\begin{Shaded}
\begin{Highlighting}[]
\FunctionTok{library}\NormalTok{(ggplot2)}
\FunctionTok{library}\NormalTok{(patchwork)}

\CommentTok{\# plotting}

\NormalTok{p1 }\OtherTok{=} \FunctionTok{ggplot}\NormalTok{(bd\_12)}\SpecialCharTok{+}
     \FunctionTok{geom\_point}\NormalTok{(}\FunctionTok{aes}\NormalTok{(}\AttributeTok{x=}\NormalTok{logload\_t, }\AttributeTok{y=}\NormalTok{logload\_tplus1, }\AttributeTok{color =} \FunctionTok{as.factor}\NormalTok{(temp)))}\SpecialCharTok{+} 
      \CommentTok{\#geom\_line(aes(x=logload\_t, y=logload\_tplus1, color="grey")) +}
     \FunctionTok{theme\_classic}\NormalTok{()  }
\CommentTok{\#p1}

\NormalTok{p2 }\OtherTok{=} \FunctionTok{ggplot}\NormalTok{(bd\_20)}\SpecialCharTok{+}
     \FunctionTok{geom\_point}\NormalTok{(}\FunctionTok{aes}\NormalTok{(}\AttributeTok{x=}\NormalTok{logload\_t, }\AttributeTok{y=}\NormalTok{logload\_tplus1, }\AttributeTok{color =} \FunctionTok{as.factor}\NormalTok{(temp)))}\SpecialCharTok{+} 
      \CommentTok{\#geom\_line(aes(x=logload\_t, y=logload\_tplus1, color="grey")) +}
     \FunctionTok{theme\_classic}\NormalTok{()  }
\CommentTok{\#p2}

\NormalTok{p1 }\SpecialCharTok{+}\NormalTok{ p2 }
\end{Highlighting}
\end{Shaded}

\pandocbounded{\includegraphics[keepaspectratio]{homework2_files/figure-latex/unnamed-chunk-2-1.pdf}}
\textbf{Answer for Q1} From these plots, we can see that the
relationship between logload\_t and logload\_tplus\_1, for both 12C and
20C, roughly forms a positive linear relationship. For 12C, the data
varies more at earlier time steps (i.e.~the data points in the lower
left are more spread out) and the points as t increases are closer
together. For the 20C data, there appears to be lower variance at early
and late time steps than in the ``middle'' time steps.

\subsubsection{Question 2}\label{question-2}

On the natural scale, we might expect Bd load to follow the
phenomenological, growth curve

\[
\mu(t + 1) = a x(t)^b
\] where \(\mu(t + 1)\) is the mean Bd load on the natural scale at time
\(t + 1\) on a frog and \(x(t)\) is the observed Bd load on the natural
scale at time \(t\). \(a\) is the per time step growth rate and
\(b < 1\) is the degree of density-dependence in Bd growth on the frog.
On the log scale, the Bd growth function is

\[
\log(\mu(t + 1)) = \log(a) + b \log(x(t))
\] which we can recognize as a linear model with intercept \(\log(a)\)
and slope \(b\). The predicted equilibrium log load on a frog
(conditional on no mortality and loss of infection) is

\[
\theta = \frac{\log(a)}{1 - b}
\] \textbf{Your goal}

Fit two Bayesian linear models, one for 12C and one for 20C, where your
predictor variable is log Bd load at time \(t\) (\(log(x(t))\) or
\texttt{logload\_t}) and your response variable is log Bd load at time
\(t + 1\) (\(log(x(t + 1))\) or \texttt{logload\_tplus1}).

For each model

\textbf{Fitting the models}

\begin{enumerate}
\def\labelenumi{\arabic{enumi}.}
\tightlist
\item
  Write out your full model using the model notation we have been
  learning in class
\item
  Discuss how you chose your prior distributions (it might be helpful to
  visually justify your prior distributions with prior prediction
  simulation plots)
\item
  Fit each of your models using a quadratic approximation (show the
  code)
\item
  Use posterior simulations and plots to test the validity of your
  fitted model (see the deer example from class as a template). Discuss
  whether your data are meeting the assumptions of your model.
\end{enumerate}

\textbf{Answers for Q2} \textbf{1. The Model}

For \(T={12,20}\),

\[
\begin{aligned}
log(x(t+1))_T &\sim \text{Normal}(log(\mu_T), \sigma) \\
log(\mu_T) &= log(a_{T}) + b_{T} [log(x(t))_T - log(\bar{x}_T)] \\
log(a_{T}) &\sim \text{Normal}(log(\bar{x}_T), 3) \\
b_{T} &\sim \text{Normal}(0, 3) \\
\sigma &\sim \text{Uniform}(0, 5)
\end{aligned}
\] where \(\bar{x}_T\) is the mean of the log load at time \(t\) for
temperature \(T\).

\textbf{2. Justification of Priors}

\(\sigma\): I have no prior knowledge of what the variance should be so
I chose an uninformative prior. (The plot of this distribution is simply
a straight line so I will neglect to show it's plot.) \(b_{T}\): This
represents the slope. I chose a weakly informative prior.
\(log(a_{T})\): This represents the y-intercept. I chose a normal
distribution with mean \(\log{\bar{x}_T}\) an a relatively large
variance to capture a reasonable range of values. \(log(\mu_T)\): I
chose this form because in the plots in part 1 there appears to be a
roughly linear relationship between the predictor variable and the
response variable. \(log(x(t+1))_T\): The chosen prior allows for the
dependence on the predictor variable and a relatively large variance.

\begin{Shaded}
\begin{Highlighting}[]
\CommentTok{\# plot for the prior of b}
\NormalTok{p\_grid }\OtherTok{=} \FunctionTok{seq}\NormalTok{(}\DecValTok{0}\NormalTok{, }\DecValTok{5}\NormalTok{, }\AttributeTok{len =} \DecValTok{1000}\NormalTok{)}
\NormalTok{dp }\OtherTok{=}\NormalTok{ p\_grid[}\DecValTok{2}\NormalTok{] }\SpecialCharTok{{-}}\NormalTok{ p\_grid[}\DecValTok{1}\NormalTok{]}
\NormalTok{prior\_b }\OtherTok{=} \FunctionTok{dunif}\NormalTok{(p\_grid, }\DecValTok{0}\NormalTok{, }\DecValTok{5}\NormalTok{) }
\NormalTok{p1 }\OtherTok{=} \FunctionTok{ggplot}\NormalTok{() }\SpecialCharTok{+} \FunctionTok{geom\_line}\NormalTok{(}\FunctionTok{aes}\NormalTok{(}\AttributeTok{x=}\NormalTok{p\_grid, }\AttributeTok{y=}\NormalTok{prior\_b}\SpecialCharTok{*}\NormalTok{dp, }\AttributeTok{linetype=}\StringTok{"Prior"}\NormalTok{)) }\SpecialCharTok{+}
     \CommentTok{\# geom\_vline(aes(xintercept = 0, colour="mean")) +}
      \FunctionTok{xlab}\NormalTok{(}\StringTok{"sigma"}\NormalTok{) }\SpecialCharTok{+}
      \FunctionTok{ylab}\NormalTok{(}\StringTok{"Probability"}\NormalTok{) }\SpecialCharTok{+} \FunctionTok{ggtitle}\NormalTok{(}\StringTok{"Temperature 12C and 20C"}\NormalTok{) }\SpecialCharTok{+} \FunctionTok{theme\_classic}\NormalTok{()}

\NormalTok{p1 }
\end{Highlighting}
\end{Shaded}

\pandocbounded{\includegraphics[keepaspectratio]{homework2_files/figure-latex/unnamed-chunk-3-1.pdf}}

\begin{Shaded}
\begin{Highlighting}[]
\CommentTok{\# plot for the prior of b}
\NormalTok{p\_grid }\OtherTok{=} \FunctionTok{seq}\NormalTok{(}\SpecialCharTok{{-}}\DecValTok{15}\NormalTok{, }\DecValTok{15}\NormalTok{, }\AttributeTok{len =} \DecValTok{1000}\NormalTok{)}
\NormalTok{dp }\OtherTok{=}\NormalTok{ p\_grid[}\DecValTok{2}\NormalTok{] }\SpecialCharTok{{-}}\NormalTok{ p\_grid[}\DecValTok{1}\NormalTok{]}
\NormalTok{prior\_b }\OtherTok{=} \FunctionTok{dnorm}\NormalTok{(p\_grid, }\DecValTok{0}\NormalTok{, }\DecValTok{3}\NormalTok{) }
\NormalTok{p1 }\OtherTok{=} \FunctionTok{ggplot}\NormalTok{() }\SpecialCharTok{+} \FunctionTok{geom\_line}\NormalTok{(}\FunctionTok{aes}\NormalTok{(}\AttributeTok{x=}\NormalTok{p\_grid, }\AttributeTok{y=}\NormalTok{prior\_b}\SpecialCharTok{*}\NormalTok{dp, }\AttributeTok{linetype=}\StringTok{"Prior"}\NormalTok{)) }\SpecialCharTok{+}
      \FunctionTok{geom\_vline}\NormalTok{(}\FunctionTok{aes}\NormalTok{(}\AttributeTok{xintercept =} \DecValTok{0}\NormalTok{, }\AttributeTok{colour=}\StringTok{"mean"}\NormalTok{)) }\SpecialCharTok{+}
      \FunctionTok{xlab}\NormalTok{(}\StringTok{"b (slope)"}\NormalTok{) }\SpecialCharTok{+}
      \FunctionTok{ylab}\NormalTok{(}\StringTok{"Probability"}\NormalTok{) }\SpecialCharTok{+} \FunctionTok{ggtitle}\NormalTok{(}\StringTok{"Temperature 12C and 20C"}\NormalTok{) }\SpecialCharTok{+} \FunctionTok{theme\_classic}\NormalTok{()}

\NormalTok{p1 }
\end{Highlighting}
\end{Shaded}

\pandocbounded{\includegraphics[keepaspectratio]{homework2_files/figure-latex/unnamed-chunk-4-1.pdf}}

\begin{Shaded}
\begin{Highlighting}[]
\CommentTok{\# plot for the prior of log(a\_\{T\})}

\DocumentationTok{\#\#\# T = 12C}

\CommentTok{\#grid for T = 12C}
\NormalTok{p\_grid\_12 }\OtherTok{=} \FunctionTok{seq}\NormalTok{(}\FunctionTok{min}\NormalTok{(bd\_12}\SpecialCharTok{$}\NormalTok{logload\_t), }\FunctionTok{max}\NormalTok{(bd\_12}\SpecialCharTok{$}\NormalTok{logload\_t), }\AttributeTok{len=}\DecValTok{100}\NormalTok{)}
\CommentTok{\#p\_grid\_12 = seq(min(bd\_12$logload\_tplus1), max(bd\_12$logload\_tplus1), len=100)}
\NormalTok{dp\_12 }\OtherTok{=}\NormalTok{ p\_grid\_12[}\DecValTok{2}\NormalTok{] }\SpecialCharTok{{-}}\NormalTok{ p\_grid\_12[}\DecValTok{1}\NormalTok{]}

\CommentTok{\# Finding mean of logload\_t at T=12}
\NormalTok{mu\_12 }\OtherTok{=} \FunctionTok{mean}\NormalTok{(bd\_12}\SpecialCharTok{$}\NormalTok{logload\_t)}

\NormalTok{prior\_log\_a\_12 }\OtherTok{=} \FunctionTok{dnorm}\NormalTok{(p\_grid\_12, mu\_12, }\DecValTok{3}\NormalTok{) }
\NormalTok{p2 }\OtherTok{=} \FunctionTok{ggplot}\NormalTok{() }\SpecialCharTok{+} \FunctionTok{geom\_line}\NormalTok{(}\FunctionTok{aes}\NormalTok{(}\AttributeTok{x=}\NormalTok{p\_grid\_12, }\AttributeTok{y=}\NormalTok{prior\_log\_a\_12}\SpecialCharTok{*}\NormalTok{dp\_12, }\AttributeTok{linetype=}\StringTok{"Prior"}\NormalTok{)) }\SpecialCharTok{+}
      \CommentTok{\#geom\_vline(aes(xintercept = 0, colour="mean")) +}
      \FunctionTok{xlab}\NormalTok{(}\StringTok{"log(a\_12) (intercept)"}\NormalTok{) }\SpecialCharTok{+}
      \FunctionTok{ylab}\NormalTok{(}\StringTok{"Probability"}\NormalTok{) }\SpecialCharTok{+} \FunctionTok{ggtitle}\NormalTok{(}\StringTok{"Temperature 12C"}\NormalTok{) }\SpecialCharTok{+} \FunctionTok{theme\_classic}\NormalTok{()}



\DocumentationTok{\#\#\# T = 20C}

\CommentTok{\#grid for T = 20C}
\NormalTok{p\_grid\_20 }\OtherTok{=} \FunctionTok{seq}\NormalTok{(}\FunctionTok{min}\NormalTok{(bd\_20}\SpecialCharTok{$}\NormalTok{logload\_t), }\FunctionTok{max}\NormalTok{(bd\_20}\SpecialCharTok{$}\NormalTok{logload\_t), }\AttributeTok{len=}\DecValTok{100}\NormalTok{)}
\NormalTok{dp\_20 }\OtherTok{=}\NormalTok{ p\_grid\_20[}\DecValTok{2}\NormalTok{] }\SpecialCharTok{{-}}\NormalTok{ p\_grid\_20[}\DecValTok{1}\NormalTok{]}

\CommentTok{\# Finding mean of logload\_t at T=12}
\NormalTok{mu\_20 }\OtherTok{=} \FunctionTok{mean}\NormalTok{(bd\_20}\SpecialCharTok{$}\NormalTok{logload\_t) }

\NormalTok{prior\_log\_a\_20 }\OtherTok{=} \FunctionTok{dnorm}\NormalTok{(p\_grid\_20, mu\_20, }\DecValTok{3}\NormalTok{) }
\NormalTok{p3 }\OtherTok{=} \FunctionTok{ggplot}\NormalTok{() }\SpecialCharTok{+} \FunctionTok{geom\_line}\NormalTok{(}\FunctionTok{aes}\NormalTok{(}\AttributeTok{x=}\NormalTok{p\_grid\_20, }\AttributeTok{y=}\NormalTok{prior\_log\_a\_20}\SpecialCharTok{*}\NormalTok{dp\_20, }\AttributeTok{linetype=}\StringTok{"Prior"}\NormalTok{)) }\SpecialCharTok{+}
      \CommentTok{\#geom\_vline(aes(xintercept = 0, colour="mean")) +}
      \FunctionTok{xlab}\NormalTok{(}\StringTok{"log(a\_20) (intercept)"}\NormalTok{) }\SpecialCharTok{+}
      \FunctionTok{ylab}\NormalTok{(}\StringTok{"Probability"}\NormalTok{) }\SpecialCharTok{+} \FunctionTok{ggtitle}\NormalTok{(}\StringTok{"Temperature 20C"}\NormalTok{) }\SpecialCharTok{+} \FunctionTok{theme\_classic}\NormalTok{()}

\NormalTok{p2}\SpecialCharTok{+}\NormalTok{p3}
\end{Highlighting}
\end{Shaded}

\pandocbounded{\includegraphics[keepaspectratio]{homework2_files/figure-latex/unnamed-chunk-5-1.pdf}}

\textbf{3. Fitting your model}

\begin{Shaded}
\begin{Highlighting}[]
\FunctionTok{library}\NormalTok{(rethinking)}
\end{Highlighting}
\end{Shaded}

\begin{verbatim}
## Loading required package: cmdstanr
\end{verbatim}

\begin{verbatim}
## This is cmdstanr version 0.8.1
\end{verbatim}

\begin{verbatim}
## - CmdStanR documentation and vignettes: mc-stan.org/cmdstanr
\end{verbatim}

\begin{verbatim}
## - CmdStan path: /Users/harkins/anaconda3/envs/r_bayesian_stats/bin/cmdstan
\end{verbatim}

\begin{verbatim}
## - CmdStan version: 2.35.0
\end{verbatim}

\begin{verbatim}
## Loading required package: posterior
\end{verbatim}

\begin{verbatim}
## This is posterior version 1.6.0
\end{verbatim}

\begin{verbatim}
## 
## Attaching package: 'posterior'
\end{verbatim}

\begin{verbatim}
## The following objects are masked from 'package:stats':
## 
##     mad, sd, var
\end{verbatim}

\begin{verbatim}
## The following objects are masked from 'package:base':
## 
##     %in%, match
\end{verbatim}

\begin{verbatim}
## Loading required package: parallel
\end{verbatim}

\begin{verbatim}
## rethinking (Version 2.40)
\end{verbatim}

\begin{verbatim}
## 
## Attaching package: 'rethinking'
\end{verbatim}

\begin{verbatim}
## The following object is masked from 'package:stats':
## 
##     rstudent
\end{verbatim}

\begin{Shaded}
\begin{Highlighting}[]
\CommentTok{\# de{-}meaning/scaling}
\CommentTok{\#bd\_data$scaled\_weight = (bd\_data$logload\_t {-} mean(bd\_data$logload\_t))}

\NormalTok{bd\_12}\SpecialCharTok{$}\NormalTok{scaled\_weight }\OtherTok{=}\NormalTok{ (bd\_12}\SpecialCharTok{$}\NormalTok{logload\_t }\SpecialCharTok{{-}} \FunctionTok{mean}\NormalTok{(bd\_12}\SpecialCharTok{$}\NormalTok{logload\_t))}
\NormalTok{bd\_20}\SpecialCharTok{$}\NormalTok{scaled\_weight }\OtherTok{=}\NormalTok{ (bd\_20}\SpecialCharTok{$}\NormalTok{logload\_t }\SpecialCharTok{{-}} \FunctionTok{mean}\NormalTok{(bd\_20}\SpecialCharTok{$}\NormalTok{logload\_t))}

\CommentTok{\# fitting the model }
\NormalTok{fit\_mod\_12 }\OtherTok{=} \FunctionTok{quap}\NormalTok{(}
             \FunctionTok{alist}\NormalTok{(}
\NormalTok{                logload\_tplus1 }\SpecialCharTok{\textasciitilde{}} \FunctionTok{dnorm}\NormalTok{(mu, sigma),}
\NormalTok{                mu }\OtherTok{\textless{}{-}}\NormalTok{ prior\_log\_a\_12 }\SpecialCharTok{+}\NormalTok{ prior\_b}\SpecialCharTok{*}\NormalTok{scaled\_weight,}
                \CommentTok{\#prior\_log\_a\_12 \textasciitilde{} dnorm(int\_12 , 3),}
\NormalTok{                prior\_log\_a\_12 }\SpecialCharTok{\textasciitilde{}} \FunctionTok{dnorm}\NormalTok{(mu\_12 , }\DecValTok{3}\NormalTok{),}
\NormalTok{                prior\_b }\SpecialCharTok{\textasciitilde{}} \FunctionTok{dnorm}\NormalTok{(}\DecValTok{0}\NormalTok{, }\DecValTok{3}\NormalTok{),}
\NormalTok{                sigma }\SpecialCharTok{\textasciitilde{}} \FunctionTok{dunif}\NormalTok{(}\DecValTok{0}\NormalTok{, }\DecValTok{5}\NormalTok{)}
\NormalTok{             ), }\AttributeTok{data =}\NormalTok{ bd\_12)}
\FunctionTok{precis}\NormalTok{(fit\_mod\_12, }\AttributeTok{prob=}\FloatTok{0.95}\NormalTok{)}
\end{Highlighting}
\end{Shaded}

\begin{verbatim}
##                     mean         sd      2.5%     97.5%
## prior_log_a_12 5.2411390 0.19210558 4.8646190 5.6176590
## prior_b        0.7736835 0.06000606 0.6560738 0.8912933
## sigma          2.0999340 0.13611842 1.8331468 2.3667212
\end{verbatim}

\begin{Shaded}
\begin{Highlighting}[]
\CommentTok{\# fitting the model }
\NormalTok{fit\_mod\_20 }\OtherTok{=} \FunctionTok{quap}\NormalTok{(}
             \FunctionTok{alist}\NormalTok{(}
\NormalTok{                logload\_tplus1 }\SpecialCharTok{\textasciitilde{}} \FunctionTok{dnorm}\NormalTok{(mu, sigma),}
\NormalTok{                mu }\OtherTok{\textless{}{-}}\NormalTok{ prior\_log\_a\_20 }\SpecialCharTok{+}\NormalTok{ prior\_b}\SpecialCharTok{*}\NormalTok{scaled\_weight,}
                \CommentTok{\#prior\_log\_a\_20 \textasciitilde{} dnorm(int\_20 , 3),}
\NormalTok{                prior\_log\_a\_20 }\SpecialCharTok{\textasciitilde{}} \FunctionTok{dnorm}\NormalTok{(mu\_20 , }\DecValTok{3}\NormalTok{),}
\NormalTok{                prior\_b }\SpecialCharTok{\textasciitilde{}} \FunctionTok{dnorm}\NormalTok{(}\DecValTok{0}\NormalTok{, }\DecValTok{3}\NormalTok{),}
\NormalTok{                sigma }\SpecialCharTok{\textasciitilde{}} \FunctionTok{dunif}\NormalTok{(}\DecValTok{0}\NormalTok{, }\DecValTok{5}\NormalTok{)}
\NormalTok{             ), }\AttributeTok{data =}\NormalTok{ bd\_20)}

\FunctionTok{precis}\NormalTok{(fit\_mod\_20, }\AttributeTok{prob=}\FloatTok{0.95}\NormalTok{)}
\end{Highlighting}
\end{Shaded}

\begin{verbatim}
##                     mean         sd      2.5%     97.5%
## prior_log_a_20 7.4800548 0.16954522 7.1477522 7.8123573
## prior_b        0.7252126 0.05851366 0.6105279 0.8398972
## sigma          1.7066349 0.12007838 1.4712856 1.9419842
\end{verbatim}

\begin{Shaded}
\begin{Highlighting}[]
\CommentTok{\# Extract the posterior}
\NormalTok{post\_12 }\OtherTok{=} \FunctionTok{extract.samples}\NormalTok{(fit\_mod\_12, }\AttributeTok{n=}\DecValTok{10000}\NormalTok{)}
\NormalTok{post\_20 }\OtherTok{=} \FunctionTok{extract.samples}\NormalTok{(fit\_mod\_20, }\AttributeTok{n=}\DecValTok{10000}\NormalTok{)}
\CommentTok{\#post\_20}

\NormalTok{diff\_log\_a }\OtherTok{=}\NormalTok{ post\_12[,}\DecValTok{1}\NormalTok{]}\SpecialCharTok{{-}}\NormalTok{post\_20[,}\DecValTok{1}\NormalTok{]}
\FunctionTok{precis}\NormalTok{(diff\_log\_a, }\AttributeTok{prob=}\FloatTok{0.95}\NormalTok{)}
\end{Highlighting}
\end{Shaded}

\begin{verbatim}
##                 mean        sd      2.5%     97.5%  histogram
## diff_log_a -2.235625 0.2550047 -2.727573 -1.736809 ▁▁▂▅▇▇▃▁▁▁
\end{verbatim}

\begin{Shaded}
\begin{Highlighting}[]
\NormalTok{diff\_b }\OtherTok{=}\NormalTok{ post\_12[,}\DecValTok{2}\NormalTok{]}\SpecialCharTok{{-}}\NormalTok{post\_20[,}\DecValTok{2}\NormalTok{]}
\FunctionTok{precis}\NormalTok{(diff\_b, }\AttributeTok{prob=}\FloatTok{0.95}\NormalTok{)}
\end{Highlighting}
\end{Shaded}

\begin{verbatim}
##              mean         sd       2.5%     97.5%     histogram
## diff_b 0.04843795 0.08379339 -0.1171129 0.2119476 ▁▁▁▂▅▇▇▅▂▁▁▁▁
\end{verbatim}

\begin{Shaded}
\begin{Highlighting}[]
\CommentTok{\# Compare prior and posterior}
\CommentTok{\#n = 10000}

\CommentTok{\# Using version 1 of the model}
\CommentTok{\# ints\_12 = rnorm(n, int\_12 , 3) \# log alpha\_T / intercepts }
\CommentTok{\# ints\_20 = rnorm(n, int\_20 , 3) \# log alpha\_T / intercepts }

\CommentTok{\# Using version 2 of the model}
\CommentTok{\#ints\_12 = rnorm(n, mu\_12 , 3) \# log alpha\_T / intercepts }
\CommentTok{\#ints\_20 = rnorm(n, mu\_20 , 3) \# log alpha\_T / intercepts }
\CommentTok{\#slopes = rnorm(n, 0, 3) \# b / slopes}

\CommentTok{\# Compare prior and posterior}
\CommentTok{\# p1 = ggplot() + geom\_density(aes(ints\_12, color=\textquotesingle{}prior\textquotesingle{})) +}
\CommentTok{\#               geom\_density(aes(post\_12[, 1], color=\textquotesingle{}posterior\textquotesingle{})) + }
\CommentTok{\#               scale\_color\_manual(values=c("blue", \textquotesingle{}black\textquotesingle{})) +}
\CommentTok{\#               xlab("log(a\_12) (intercept)") + }
\CommentTok{\#               ylab("Density") +}
\CommentTok{\#         ggtitle("Temperature 12C") +}
\CommentTok{\#               theme\_classic()}
\CommentTok{\# }
\CommentTok{\# p2 = ggplot() + geom\_density(aes(ints\_20, color=\textquotesingle{}prior\textquotesingle{})) +}
\CommentTok{\#               geom\_density(aes(post\_20[, 1], color=\textquotesingle{}posterior\textquotesingle{})) + }
\CommentTok{\#               scale\_color\_manual(values=c("blue", \textquotesingle{}black\textquotesingle{})) +}
\CommentTok{\#               xlab("log(a\_20) (intercept)") + }
\CommentTok{\#               ylab("Density") +}
\CommentTok{\#         ggtitle("Temperature 20C") +}
\CommentTok{\#               theme\_classic()}
\CommentTok{\# }
\CommentTok{\# }
\CommentTok{\# p3 = ggplot() + geom\_density(aes(slopes, color="prior")) +}
\CommentTok{\#               geom\_density(aes(post\_12[, 2], color="posterior")) +}
\CommentTok{\#               scale\_color\_manual(values=c("blue", \textquotesingle{}black\textquotesingle{})) +}
\CommentTok{\#               xlab("b (slope)") + }
\CommentTok{\#               ylab("Density") +}
\CommentTok{\#               theme\_classic()}
\CommentTok{\# }
\CommentTok{\# p1+p2+p3}
\end{Highlighting}
\end{Shaded}

\begin{Shaded}
\begin{Highlighting}[]
\CommentTok{\# Just posterior}
\NormalTok{p1 }\OtherTok{=} \FunctionTok{ggplot}\NormalTok{() }\SpecialCharTok{+} 
                \FunctionTok{geom\_density}\NormalTok{(}\FunctionTok{aes}\NormalTok{(post\_12[, }\DecValTok{1}\NormalTok{], }\AttributeTok{color=}\StringTok{\textquotesingle{}posterior\textquotesingle{}}\NormalTok{)) }\SpecialCharTok{+} 
                \FunctionTok{scale\_color\_manual}\NormalTok{(}\AttributeTok{values=}\FunctionTok{c}\NormalTok{(}\StringTok{"blue"}\NormalTok{, }\StringTok{\textquotesingle{}black\textquotesingle{}}\NormalTok{)) }\SpecialCharTok{+}
                \FunctionTok{xlab}\NormalTok{(}\StringTok{"log(a\_12) (intercept)"}\NormalTok{) }\SpecialCharTok{+} 
                \FunctionTok{ylab}\NormalTok{(}\StringTok{"Density"}\NormalTok{) }\SpecialCharTok{+}
                \FunctionTok{theme\_classic}\NormalTok{()}
\NormalTok{p2 }\OtherTok{=} \FunctionTok{ggplot}\NormalTok{() }\SpecialCharTok{+} 
                \FunctionTok{geom\_density}\NormalTok{(}\FunctionTok{aes}\NormalTok{(post\_20[, }\DecValTok{1}\NormalTok{], }\AttributeTok{color=}\StringTok{\textquotesingle{}posterior\textquotesingle{}}\NormalTok{)) }\SpecialCharTok{+} 
                \FunctionTok{scale\_color\_manual}\NormalTok{(}\AttributeTok{values=}\FunctionTok{c}\NormalTok{(}\StringTok{"blue"}\NormalTok{, }\StringTok{\textquotesingle{}black\textquotesingle{}}\NormalTok{)) }\SpecialCharTok{+}
                \FunctionTok{xlab}\NormalTok{(}\StringTok{"log(a\_20) (intercept)"}\NormalTok{) }\SpecialCharTok{+} 
                \FunctionTok{ylab}\NormalTok{(}\StringTok{"Density"}\NormalTok{) }\SpecialCharTok{+}
                \FunctionTok{theme\_classic}\NormalTok{()}

\NormalTok{p3 }\OtherTok{=} \FunctionTok{ggplot}\NormalTok{() }\SpecialCharTok{+}
                \FunctionTok{geom\_density}\NormalTok{(}\FunctionTok{aes}\NormalTok{(post\_12[, }\DecValTok{2}\NormalTok{], }\AttributeTok{color=}\StringTok{"posterior"}\NormalTok{)) }\SpecialCharTok{+}
                \FunctionTok{scale\_color\_manual}\NormalTok{(}\AttributeTok{values=}\FunctionTok{c}\NormalTok{(}\StringTok{"blue"}\NormalTok{, }\StringTok{\textquotesingle{}black\textquotesingle{}}\NormalTok{)) }\SpecialCharTok{+}
                \FunctionTok{xlab}\NormalTok{(}\StringTok{"b(slope) T = 12C"}\NormalTok{) }\SpecialCharTok{+} 
                \FunctionTok{ylab}\NormalTok{(}\StringTok{"Density"}\NormalTok{) }\SpecialCharTok{+}
                \FunctionTok{theme\_classic}\NormalTok{()}

\NormalTok{p4 }\OtherTok{=} \FunctionTok{ggplot}\NormalTok{() }\SpecialCharTok{+}
                \FunctionTok{geom\_density}\NormalTok{(}\FunctionTok{aes}\NormalTok{(post\_20[, }\DecValTok{2}\NormalTok{], }\AttributeTok{color=}\StringTok{"posterior"}\NormalTok{)) }\SpecialCharTok{+}
                \FunctionTok{scale\_color\_manual}\NormalTok{(}\AttributeTok{values=}\FunctionTok{c}\NormalTok{(}\StringTok{"blue"}\NormalTok{, }\StringTok{\textquotesingle{}black\textquotesingle{}}\NormalTok{)) }\SpecialCharTok{+}
                \FunctionTok{xlab}\NormalTok{(}\StringTok{"b(slope) T = 20C"}\NormalTok{) }\SpecialCharTok{+} 
                \FunctionTok{ylab}\NormalTok{(}\StringTok{"Density"}\NormalTok{) }\SpecialCharTok{+}
                \FunctionTok{theme\_classic}\NormalTok{()}
\NormalTok{p1}\SpecialCharTok{+}\NormalTok{p2}\SpecialCharTok{+}\NormalTok{p3}\SpecialCharTok{+}\NormalTok{p4}
\end{Highlighting}
\end{Shaded}

\pandocbounded{\includegraphics[keepaspectratio]{homework2_files/figure-latex/unnamed-chunk-7-1.pdf}}

\begin{Shaded}
\begin{Highlighting}[]
\CommentTok{\# posterior predictions }
\FunctionTok{library}\NormalTok{(data.table)}
\end{Highlighting}
\end{Shaded}

\begin{verbatim}
## 
## Attaching package: 'data.table'
\end{verbatim}

\begin{verbatim}
## The following objects are masked from 'package:dplyr':
## 
##     between, first, last
\end{verbatim}

\begin{Shaded}
\begin{Highlighting}[]
\NormalTok{vals }\OtherTok{=} \FunctionTok{seq}\NormalTok{(}\FunctionTok{min}\NormalTok{(bd\_12}\SpecialCharTok{$}\NormalTok{scaled\_weight, bd\_20}\SpecialCharTok{$}\NormalTok{scaled\_weight), }\FunctionTok{max}\NormalTok{(bd\_12}\SpecialCharTok{$}\NormalTok{scaled\_weight, bd\_20}\SpecialCharTok{$}\NormalTok{scaled\_weight), }\AttributeTok{len=}\DecValTok{50}\NormalTok{)}

\CommentTok{\# For T = 12}
\NormalTok{lines\_12}\OtherTok{=} \FunctionTok{sapply}\NormalTok{(vals, }\ControlFlowTok{function}\NormalTok{(x) post\_12[}\DecValTok{1}\SpecialCharTok{:}\DecValTok{100}\NormalTok{, }\DecValTok{1}\NormalTok{] }\SpecialCharTok{+}\NormalTok{ x}\SpecialCharTok{*}\NormalTok{post\_12[}\DecValTok{1}\SpecialCharTok{:}\DecValTok{100}\NormalTok{, }\DecValTok{2}\NormalTok{])}
\NormalTok{lines\_12\_dt }\OtherTok{=} \FunctionTok{data.table}\NormalTok{(}\FunctionTok{data.frame}\NormalTok{(}\FunctionTok{t}\NormalTok{(lines\_12)))}
\NormalTok{lines\_12\_dt}\SpecialCharTok{$}\NormalTok{scaled\_weight }\OtherTok{=}\NormalTok{ vals}
\NormalTok{lines\_12\_melt }\OtherTok{=} \FunctionTok{melt}\NormalTok{(lines\_12\_dt, }\AttributeTok{id.vars=}\StringTok{"scaled\_weight"}\NormalTok{)}

\NormalTok{lines\_12\_plt }\OtherTok{=} \FunctionTok{ggplot}\NormalTok{(lines\_12\_melt) }\SpecialCharTok{+} \FunctionTok{geom\_line}\NormalTok{(}\FunctionTok{aes}\NormalTok{(}\AttributeTok{x=}\NormalTok{scaled\_weight, }\AttributeTok{y=}\NormalTok{value, }\AttributeTok{group=}\NormalTok{variable), }\AttributeTok{alpha=}\FloatTok{0.2}\NormalTok{) }\SpecialCharTok{+}
                          \FunctionTok{theme\_classic}\NormalTok{() }\SpecialCharTok{+} \FunctionTok{xlab}\NormalTok{(}\StringTok{"logload\_t"}\NormalTok{) }\SpecialCharTok{+}
                          \FunctionTok{ylab}\NormalTok{(}\StringTok{"logload\_tplus\_1"}\NormalTok{) }\SpecialCharTok{+}
              \FunctionTok{ggtitle}\NormalTok{(}\StringTok{"Temperature 12C"}\NormalTok{)}


\CommentTok{\# For T = 20}
\NormalTok{lines\_20}\OtherTok{=} \FunctionTok{sapply}\NormalTok{(vals, }\ControlFlowTok{function}\NormalTok{(x) post\_20[}\DecValTok{1}\SpecialCharTok{:}\DecValTok{100}\NormalTok{, }\DecValTok{1}\NormalTok{] }\SpecialCharTok{+}\NormalTok{ x}\SpecialCharTok{*}\NormalTok{post\_20[}\DecValTok{1}\SpecialCharTok{:}\DecValTok{100}\NormalTok{, }\DecValTok{2}\NormalTok{])}
\NormalTok{lines\_20\_dt }\OtherTok{=} \FunctionTok{data.table}\NormalTok{(}\FunctionTok{data.frame}\NormalTok{(}\FunctionTok{t}\NormalTok{(lines\_20)))}
\NormalTok{lines\_20\_dt}\SpecialCharTok{$}\NormalTok{scaled\_weight }\OtherTok{=}\NormalTok{ vals}
\NormalTok{lines\_20\_melt }\OtherTok{=} \FunctionTok{melt}\NormalTok{(lines\_20\_dt, }\AttributeTok{id.vars=}\StringTok{"scaled\_weight"}\NormalTok{)}

\NormalTok{lines\_20\_plt }\OtherTok{=} \FunctionTok{ggplot}\NormalTok{(lines\_20\_melt) }\SpecialCharTok{+} \FunctionTok{geom\_line}\NormalTok{(}\FunctionTok{aes}\NormalTok{(}\AttributeTok{x=}\NormalTok{scaled\_weight, }\AttributeTok{y=}\NormalTok{value, }\AttributeTok{group=}\NormalTok{variable), }\AttributeTok{alpha=}\FloatTok{0.2}\NormalTok{) }\SpecialCharTok{+}
                          \FunctionTok{theme\_classic}\NormalTok{() }\SpecialCharTok{+} \FunctionTok{xlab}\NormalTok{(}\StringTok{"logload\_t"}\NormalTok{) }\SpecialCharTok{+}
                          \FunctionTok{ylab}\NormalTok{(}\StringTok{"logload\_tplus\_1"}\NormalTok{)}\SpecialCharTok{+}
              \FunctionTok{ggtitle}\NormalTok{(}\StringTok{"Temperature 20C"}\NormalTok{)}

\NormalTok{lines\_12\_plt }\SpecialCharTok{+}\NormalTok{ lines\_20\_plt}
\end{Highlighting}
\end{Shaded}

\pandocbounded{\includegraphics[keepaspectratio]{homework2_files/figure-latex/unnamed-chunk-8-1.pdf}}

\textbf{4. Viability of Model}

\begin{Shaded}
\begin{Highlighting}[]
\CommentTok{\# Generate mean predictions}

\CommentTok{\# Step 1: Draw beta0 and beta1}
\NormalTok{post\_12 }\OtherTok{=} \FunctionTok{extract.samples}\NormalTok{(fit\_mod\_12, }\AttributeTok{n=}\DecValTok{10000}\NormalTok{)}
\NormalTok{prior\_log\_a\_12 }\OtherTok{=}\NormalTok{ post\_12[, }\DecValTok{1}\NormalTok{]}
\NormalTok{b }\OtherTok{=}\NormalTok{ post\_12[, }\DecValTok{2}\NormalTok{]}
\CommentTok{\#sigma = post\_12[, 3]}

\CommentTok{\# Step 2{-}4: For a doe body weight, calculate mean}
\NormalTok{pllt12 }\OtherTok{=} \FunctionTok{seq}\NormalTok{(}\FunctionTok{min}\NormalTok{(bd\_12}\SpecialCharTok{$}\NormalTok{logload\_t), }\FunctionTok{max}\NormalTok{(bd\_12}\SpecialCharTok{$}\NormalTok{logload\_t), }\AttributeTok{len=}\DecValTok{50}\NormalTok{)}
\NormalTok{pllt12\_scaled }\OtherTok{=}\NormalTok{ pllt12 }\SpecialCharTok{{-}} \FunctionTok{mean}\NormalTok{(bd\_12}\SpecialCharTok{$}\NormalTok{logload\_t)}
\NormalTok{llt12\_mean }\OtherTok{=} \FunctionTok{sapply}\NormalTok{(pllt12\_scaled, }\ControlFlowTok{function}\NormalTok{(x) prior\_log\_a\_12 }\SpecialCharTok{+}\NormalTok{ x}\SpecialCharTok{*}\NormalTok{b)}

\CommentTok{\# Step 5. Summarize}
\NormalTok{median\_llt12 }\OtherTok{=} \FunctionTok{apply}\NormalTok{(llt12\_mean, }\DecValTok{2}\NormalTok{, median)}
\NormalTok{lowerupper\_pred }\OtherTok{=} \FunctionTok{apply}\NormalTok{(llt12\_mean, }\DecValTok{2}\NormalTok{, }\ControlFlowTok{function}\NormalTok{(x) }\FunctionTok{quantile}\NormalTok{(x, }\FunctionTok{c}\NormalTok{(}\FloatTok{0.025}\NormalTok{, }\FloatTok{0.975}\NormalTok{)))}
\NormalTok{llt12\_df }\OtherTok{=} \FunctionTok{data.frame}\NormalTok{(}\AttributeTok{med=}\NormalTok{median\_llt12, }
                     \AttributeTok{lower=}\NormalTok{lowerupper\_pred[}\DecValTok{1}\NormalTok{, ], }
                     \AttributeTok{upper=}\NormalTok{lowerupper\_pred[}\DecValTok{2}\NormalTok{, ],}
                     \AttributeTok{logload\_t=}\NormalTok{pllt12)}
\CommentTok{\# Step 6. Visualize}
\NormalTok{p1 }\OtherTok{=} \FunctionTok{ggplot}\NormalTok{() }\SpecialCharTok{+} \FunctionTok{geom\_point}\NormalTok{(}\AttributeTok{data=}\NormalTok{bd\_12, }\FunctionTok{aes}\NormalTok{(}\AttributeTok{x=}\NormalTok{logload\_t, }\AttributeTok{y=}\NormalTok{logload\_tplus1, }\AttributeTok{color=}\StringTok{"Observed"}\NormalTok{)) }\SpecialCharTok{+}
           \FunctionTok{geom\_line}\NormalTok{(}\AttributeTok{data=}\NormalTok{llt12\_df, }\FunctionTok{aes}\NormalTok{(}\AttributeTok{x=}\NormalTok{pllt12, }\AttributeTok{y=}\NormalTok{med, }\AttributeTok{color=}\StringTok{"Mean prediction"}\NormalTok{)) }\SpecialCharTok{+}
           \FunctionTok{geom\_ribbon}\NormalTok{(}\AttributeTok{data=}\NormalTok{llt12\_df, }\FunctionTok{aes}\NormalTok{(}\AttributeTok{x=}\NormalTok{pllt12, }\AttributeTok{ymin=}\NormalTok{lower, }\AttributeTok{ymax=}\NormalTok{upper, }\AttributeTok{fill=}\StringTok{"95\% CI}\SpecialCharTok{\textbackslash{}n}\StringTok{around mean"}\NormalTok{),}
                                         \AttributeTok{alpha=}\FloatTok{0.2}\NormalTok{) }\SpecialCharTok{+}
           \FunctionTok{scale\_color\_manual}\NormalTok{(}\AttributeTok{values=}\FunctionTok{c}\NormalTok{(}\StringTok{"blue"}\NormalTok{, }\StringTok{\textquotesingle{}black\textquotesingle{}}\NormalTok{)) }\SpecialCharTok{+}
           \FunctionTok{scale\_fill\_manual}\NormalTok{(}\AttributeTok{values=}\FunctionTok{c}\NormalTok{(}\StringTok{\textquotesingle{}blue\textquotesingle{}}\NormalTok{)) }\SpecialCharTok{+}
           \FunctionTok{theme\_classic}\NormalTok{() }\SpecialCharTok{+}
           \FunctionTok{xlab}\NormalTok{(}\StringTok{"logload\_t"}\NormalTok{) }\SpecialCharTok{+}
       \FunctionTok{ylab}\NormalTok{(}\StringTok{"logload\_tplus1"}\NormalTok{) }\SpecialCharTok{+}
       \FunctionTok{ggtitle}\NormalTok{(}\StringTok{"Temperature 12 C"}\NormalTok{)}


\CommentTok{\# p1 = ggplot() + geom\_point(data=bd\_12, aes(x=scaled\_weight, y=logload\_tplus1, color="Observed")) +}
\CommentTok{\#          geom\_line(data=llt12\_df, aes(x=pllt12\_scaled, y=med, color="Mean prediction")) + }
\CommentTok{\#          geom\_ribbon(data=llt12\_df, aes(x=pllt12\_scaled, ymin=lower, ymax=upper, fill="95\% CI\textbackslash{}naround mean"), }
\CommentTok{\#                                        alpha=0.2) +}
\CommentTok{\#          scale\_color\_manual(values=c("blue", \textquotesingle{}black\textquotesingle{})) +}
\CommentTok{\#          scale\_fill\_manual(values=c(\textquotesingle{}blue\textquotesingle{})) +}
\CommentTok{\#          theme\_classic() +}
\CommentTok{\#          xlab("de{-}meaned logload\_t") + }
\CommentTok{\#        ylab("logload\_tplus1") +}
\CommentTok{\#        ggtitle("Temperature 12 C")}


\DocumentationTok{\#\#\#\# Repeating the same process for temperature 20C}
\NormalTok{post\_20 }\OtherTok{=} \FunctionTok{extract.samples}\NormalTok{(fit\_mod\_20, }\AttributeTok{n=}\DecValTok{10000}\NormalTok{)}
\NormalTok{prior\_log\_a\_20 }\OtherTok{=}\NormalTok{ post\_20[, }\DecValTok{1}\NormalTok{]}
\NormalTok{b }\OtherTok{=}\NormalTok{ post\_20[, }\DecValTok{2}\NormalTok{]}
\CommentTok{\#sigma = post\_20[, 3]}

\CommentTok{\# Step 2{-}4: For a doe body weight, calculate mean}
\NormalTok{pllt20 }\OtherTok{=} \FunctionTok{seq}\NormalTok{(}\FunctionTok{min}\NormalTok{(bd\_20}\SpecialCharTok{$}\NormalTok{logload\_t), }\FunctionTok{max}\NormalTok{(bd\_20}\SpecialCharTok{$}\NormalTok{logload\_t), }\AttributeTok{len=}\DecValTok{50}\NormalTok{)}
\NormalTok{pllt20\_scaled }\OtherTok{=}\NormalTok{ pllt20 }\SpecialCharTok{{-}} \FunctionTok{mean}\NormalTok{(bd\_20}\SpecialCharTok{$}\NormalTok{logload\_t)}
\NormalTok{llt20\_mean }\OtherTok{=} \FunctionTok{sapply}\NormalTok{(pllt20\_scaled, }\ControlFlowTok{function}\NormalTok{(x) prior\_log\_a\_20 }\SpecialCharTok{+}\NormalTok{ x}\SpecialCharTok{*}\NormalTok{b)}

\CommentTok{\# Step 5. Summarize}
\NormalTok{median\_llt20 }\OtherTok{=} \FunctionTok{apply}\NormalTok{(llt20\_mean, }\DecValTok{2}\NormalTok{, median)}
\NormalTok{lowerupper\_pred }\OtherTok{=} \FunctionTok{apply}\NormalTok{(llt20\_mean, }\DecValTok{2}\NormalTok{, }\ControlFlowTok{function}\NormalTok{(x) }\FunctionTok{quantile}\NormalTok{(x, }\FunctionTok{c}\NormalTok{(}\FloatTok{0.025}\NormalTok{, }\FloatTok{0.975}\NormalTok{)))}
\NormalTok{llt20\_df }\OtherTok{=} \FunctionTok{data.frame}\NormalTok{(}\AttributeTok{med=}\NormalTok{median\_llt20, }
                     \AttributeTok{lower=}\NormalTok{lowerupper\_pred[}\DecValTok{1}\NormalTok{, ], }
                     \AttributeTok{upper=}\NormalTok{lowerupper\_pred[}\DecValTok{2}\NormalTok{, ],}
                     \AttributeTok{logload\_t=}\NormalTok{pllt12)}
\CommentTok{\#llt20\_df}

\CommentTok{\# Step 6. Visualize}
\NormalTok{p2 }\OtherTok{=} \FunctionTok{ggplot}\NormalTok{() }\SpecialCharTok{+} \FunctionTok{geom\_point}\NormalTok{(}\AttributeTok{data=}\NormalTok{bd\_20, }\FunctionTok{aes}\NormalTok{(}\AttributeTok{x=}\NormalTok{logload\_t, }\AttributeTok{y=}\NormalTok{logload\_tplus1, }\AttributeTok{color=}\StringTok{"Observed"}\NormalTok{)) }\SpecialCharTok{+}
           \FunctionTok{geom\_line}\NormalTok{(}\AttributeTok{data=}\NormalTok{llt20\_df, }\FunctionTok{aes}\NormalTok{(}\AttributeTok{x=}\NormalTok{pllt20, }\AttributeTok{y=}\NormalTok{med, }\AttributeTok{color=}\StringTok{"Mean prediction"}\NormalTok{)) }\SpecialCharTok{+}
           \FunctionTok{geom\_ribbon}\NormalTok{(}\AttributeTok{data=}\NormalTok{llt20\_df, }\FunctionTok{aes}\NormalTok{(}\AttributeTok{x=}\NormalTok{pllt20, }\AttributeTok{ymin=}\NormalTok{lower, }\AttributeTok{ymax=}\NormalTok{upper, }\AttributeTok{fill=}\StringTok{"95\% CI}\SpecialCharTok{\textbackslash{}n}\StringTok{around mean"}\NormalTok{),}
                                         \AttributeTok{alpha=}\FloatTok{0.2}\NormalTok{) }\SpecialCharTok{+}
           \FunctionTok{scale\_color\_manual}\NormalTok{(}\AttributeTok{values=}\FunctionTok{c}\NormalTok{(}\StringTok{"blue"}\NormalTok{, }\StringTok{\textquotesingle{}black\textquotesingle{}}\NormalTok{)) }\SpecialCharTok{+}
           \FunctionTok{scale\_fill\_manual}\NormalTok{(}\AttributeTok{values=}\FunctionTok{c}\NormalTok{(}\StringTok{\textquotesingle{}blue\textquotesingle{}}\NormalTok{)) }\SpecialCharTok{+}
           \FunctionTok{theme\_classic}\NormalTok{() }\SpecialCharTok{+}
           \FunctionTok{xlab}\NormalTok{(}\StringTok{"logload\_t"}\NormalTok{) }\SpecialCharTok{+}
       \FunctionTok{ylab}\NormalTok{(}\StringTok{"logload\_tplus1"}\NormalTok{) }\SpecialCharTok{+}
       \FunctionTok{ggtitle}\NormalTok{(}\StringTok{"Temperature 20 C"}\NormalTok{)}

\CommentTok{\# p2 = ggplot() + geom\_point(data=bd\_20, aes(x=scaled\_weight, y=logload\_tplus1, color="Observed")) +}
\CommentTok{\#          geom\_line(data=llt20\_df, aes(x=pllt20\_scaled, y=med, color="Mean prediction")) + }
\CommentTok{\#          geom\_ribbon(data=llt20\_df, aes(x=pllt20\_scaled, ymin=lower, ymax=upper, fill="95\% CI\textbackslash{}naround mean"), }
\CommentTok{\#                                        alpha=0.2) +}
\CommentTok{\#          scale\_color\_manual(values=c("blue", \textquotesingle{}black\textquotesingle{})) +}
\CommentTok{\#          scale\_fill\_manual(values=c(\textquotesingle{}blue\textquotesingle{})) +}
\CommentTok{\#          theme\_classic() +}
\CommentTok{\#          xlab("de{-}meaned logload\_t") + }
\CommentTok{\#        ylab("logload\_tplus1") +}
\CommentTok{\#        ggtitle("Temperature 20 C")}

\NormalTok{p1 }\SpecialCharTok{+}\NormalTok{ p2}
\end{Highlighting}
\end{Shaded}

\pandocbounded{\includegraphics[keepaspectratio]{homework2_files/figure-latex/unnamed-chunk-9-1.pdf}}

\begin{Shaded}
\begin{Highlighting}[]
\CommentTok{\# Step 1: Draw beta0 and beta1}
\NormalTok{n }\OtherTok{=} \DecValTok{10000}
\NormalTok{post\_12 }\OtherTok{=} \FunctionTok{extract.samples}\NormalTok{(fit\_mod\_12, }\AttributeTok{n=}\DecValTok{10000}\NormalTok{)}
\NormalTok{prior\_log\_a\_12 }\OtherTok{=}\NormalTok{ post\_12[, }\DecValTok{1}\NormalTok{]}
\NormalTok{b }\OtherTok{=}\NormalTok{ post\_12[, }\DecValTok{2}\NormalTok{]}
\NormalTok{sigma }\OtherTok{=}\NormalTok{ post\_12[, }\DecValTok{3}\NormalTok{]}

\CommentTok{\# Step 2{-}4: For a doe body weight, calculate mean and draw from normal}
\NormalTok{llt12\_distribution }\OtherTok{=} \FunctionTok{sapply}\NormalTok{(pllt12\_scaled, }\ControlFlowTok{function}\NormalTok{(x) }\FunctionTok{rnorm}\NormalTok{(n, prior\_log\_a\_12 }\SpecialCharTok{+}\NormalTok{ x}\SpecialCharTok{*}\NormalTok{b, sigma)) }\CommentTok{\# drawing random logload\_tplus1}

\CommentTok{\# Step 5. Summarize}
\NormalTok{median\_llt12\_dist }\OtherTok{=} \FunctionTok{apply}\NormalTok{(llt12\_distribution, }\DecValTok{2}\NormalTok{, median)}
\NormalTok{lowerupper\_llt12\_dist }\OtherTok{=} \FunctionTok{apply}\NormalTok{(llt12\_distribution, }\DecValTok{2}\NormalTok{, }\ControlFlowTok{function}\NormalTok{(x) }\FunctionTok{quantile}\NormalTok{(x, }\FunctionTok{c}\NormalTok{(}\FloatTok{0.025}\NormalTok{, }\FloatTok{0.975}\NormalTok{)))}
\NormalTok{llt12\_dist\_df }\OtherTok{=} \FunctionTok{data.frame}\NormalTok{(}\AttributeTok{med=}\NormalTok{median\_llt12\_dist, }
                     \AttributeTok{lower=}\NormalTok{lowerupper\_llt12\_dist[}\DecValTok{1}\NormalTok{, ], }
                     \AttributeTok{upper=}\NormalTok{lowerupper\_llt12\_dist[}\DecValTok{2}\NormalTok{, ],}
                     \AttributeTok{logload\_t=}\NormalTok{pllt12)}

\NormalTok{p1 }\OtherTok{=} \FunctionTok{ggplot}\NormalTok{() }\SpecialCharTok{+} \FunctionTok{geom\_point}\NormalTok{(}\AttributeTok{data=}\NormalTok{bd\_12, }\FunctionTok{aes}\NormalTok{(}\AttributeTok{x=}\NormalTok{logload\_t, }\AttributeTok{y=}\NormalTok{logload\_tplus1, }\AttributeTok{color=}\StringTok{"Observed"}\NormalTok{)) }\SpecialCharTok{+}
           \FunctionTok{geom\_line}\NormalTok{(}\AttributeTok{data=}\NormalTok{llt20\_df, }\FunctionTok{aes}\NormalTok{(}\AttributeTok{x=}\NormalTok{pllt12, }\AttributeTok{y=}\NormalTok{med, }\AttributeTok{color=}\StringTok{"Mean prediction"}\NormalTok{)) }\SpecialCharTok{+} 
           \FunctionTok{geom\_ribbon}\NormalTok{(}\AttributeTok{data=}\NormalTok{llt20\_df, }\FunctionTok{aes}\NormalTok{(}\AttributeTok{x=}\NormalTok{pllt12, }\AttributeTok{ymin=}\NormalTok{lower, }\AttributeTok{ymax=}\NormalTok{upper, }\AttributeTok{fill=}\StringTok{"95\% CI}\SpecialCharTok{\textbackslash{}n}\StringTok{around mean"}\NormalTok{), }
                                         \AttributeTok{alpha=}\FloatTok{0.2}\NormalTok{) }\SpecialCharTok{+}
           \FunctionTok{geom\_ribbon}\NormalTok{(}\AttributeTok{data=}\NormalTok{llt12\_dist\_df, }\FunctionTok{aes}\NormalTok{(}\AttributeTok{x=}\NormalTok{pllt12, }\AttributeTok{ymin=}\NormalTok{lower, }\AttributeTok{ymax=}\NormalTok{upper, }\AttributeTok{fill=}\StringTok{"95\% CI}\SpecialCharTok{\textbackslash{}n}\StringTok{around prediction"}\NormalTok{), }
                                         \AttributeTok{alpha=}\FloatTok{0.2}\NormalTok{) }\SpecialCharTok{+}
           \FunctionTok{scale\_color\_manual}\NormalTok{(}\AttributeTok{values=}\FunctionTok{c}\NormalTok{(}\StringTok{"blue"}\NormalTok{, }\StringTok{\textquotesingle{}black\textquotesingle{}}\NormalTok{)) }\SpecialCharTok{+}
           \FunctionTok{scale\_fill\_manual}\NormalTok{(}\AttributeTok{values=}\FunctionTok{c}\NormalTok{(}\StringTok{\textquotesingle{}blue\textquotesingle{}}\NormalTok{, }\StringTok{\textquotesingle{}red\textquotesingle{}}\NormalTok{)) }\SpecialCharTok{+}
           \FunctionTok{theme\_classic}\NormalTok{() }\SpecialCharTok{+}
           \FunctionTok{xlab}\NormalTok{(}\StringTok{"logload\_t"}\NormalTok{) }\SpecialCharTok{+} \FunctionTok{ylab}\NormalTok{(}\StringTok{"logload\_tplus1"}\NormalTok{) }\SpecialCharTok{+} \FunctionTok{ggtitle}\NormalTok{(}\StringTok{" Temperature = 12C"}\NormalTok{)}

\CommentTok{\# Repeating these steps for T = 20C }
\CommentTok{\# Step 1: Draw beta0 and beta1}
\NormalTok{post\_20 }\OtherTok{=} \FunctionTok{extract.samples}\NormalTok{(fit\_mod\_20, }\AttributeTok{n=}\DecValTok{10000}\NormalTok{)}
\NormalTok{prior\_log\_a\_20 }\OtherTok{=}\NormalTok{ post\_20[, }\DecValTok{1}\NormalTok{]}
\NormalTok{b }\OtherTok{=}\NormalTok{ post\_20[, }\DecValTok{2}\NormalTok{]}
\NormalTok{sigma }\OtherTok{=}\NormalTok{ post\_20[, }\DecValTok{3}\NormalTok{]}

\CommentTok{\# Step 2{-}4: For a doe body weight, calculate mean and draw from normal}
\NormalTok{llt20\_distribution }\OtherTok{=} \FunctionTok{sapply}\NormalTok{(pllt20\_scaled, }\ControlFlowTok{function}\NormalTok{(x) }\FunctionTok{rnorm}\NormalTok{(n, prior\_log\_a\_20 }\SpecialCharTok{+}\NormalTok{ x}\SpecialCharTok{*}\NormalTok{b, sigma)) }\CommentTok{\# drawing random logload\_tplus1}

\CommentTok{\# Step 5. Summarize}
\NormalTok{median\_llt20\_dist }\OtherTok{=} \FunctionTok{apply}\NormalTok{(llt20\_distribution, }\DecValTok{2}\NormalTok{, median)}
\NormalTok{lowerupper\_llt20\_dist }\OtherTok{=} \FunctionTok{apply}\NormalTok{(llt20\_distribution, }\DecValTok{2}\NormalTok{, }\ControlFlowTok{function}\NormalTok{(x) }\FunctionTok{quantile}\NormalTok{(x, }\FunctionTok{c}\NormalTok{(}\FloatTok{0.025}\NormalTok{, }\FloatTok{0.975}\NormalTok{)))}
\NormalTok{llt20\_dist\_df }\OtherTok{=} \FunctionTok{data.frame}\NormalTok{(}\AttributeTok{med=}\NormalTok{median\_llt20\_dist, }
                     \AttributeTok{lower=}\NormalTok{lowerupper\_llt20\_dist[}\DecValTok{1}\NormalTok{, ], }
                     \AttributeTok{upper=}\NormalTok{lowerupper\_llt20\_dist[}\DecValTok{2}\NormalTok{, ],}
                     \AttributeTok{logload\_t=}\NormalTok{pllt20)}

\NormalTok{p2 }\OtherTok{=} \FunctionTok{ggplot}\NormalTok{() }\SpecialCharTok{+} \FunctionTok{geom\_point}\NormalTok{(}\AttributeTok{data=}\NormalTok{bd\_20, }\FunctionTok{aes}\NormalTok{(}\AttributeTok{x=}\NormalTok{logload\_t, }\AttributeTok{y=}\NormalTok{logload\_tplus1, }\AttributeTok{color=}\StringTok{"Observed"}\NormalTok{)) }\SpecialCharTok{+}
           \FunctionTok{geom\_line}\NormalTok{(}\AttributeTok{data=}\NormalTok{llt20\_df, }\FunctionTok{aes}\NormalTok{(}\AttributeTok{x=}\NormalTok{pllt20, }\AttributeTok{y=}\NormalTok{med, }\AttributeTok{color=}\StringTok{"Mean prediction"}\NormalTok{)) }\SpecialCharTok{+} 
           \FunctionTok{geom\_ribbon}\NormalTok{(}\AttributeTok{data=}\NormalTok{llt20\_df, }\FunctionTok{aes}\NormalTok{(}\AttributeTok{x=}\NormalTok{pllt20, }\AttributeTok{ymin=}\NormalTok{lower, }\AttributeTok{ymax=}\NormalTok{upper, }\AttributeTok{fill=}\StringTok{"95\% CI}\SpecialCharTok{\textbackslash{}n}\StringTok{around mean"}\NormalTok{), }
                                         \AttributeTok{alpha=}\FloatTok{0.2}\NormalTok{) }\SpecialCharTok{+}
           \FunctionTok{geom\_ribbon}\NormalTok{(}\AttributeTok{data=}\NormalTok{llt20\_dist\_df, }\FunctionTok{aes}\NormalTok{(}\AttributeTok{x=}\NormalTok{pllt20, }\AttributeTok{ymin=}\NormalTok{lower, }\AttributeTok{ymax=}\NormalTok{upper, }\AttributeTok{fill=}\StringTok{"95\% CI}\SpecialCharTok{\textbackslash{}n}\StringTok{around prediction"}\NormalTok{), }
                                         \AttributeTok{alpha=}\FloatTok{0.2}\NormalTok{) }\SpecialCharTok{+}
           \FunctionTok{scale\_color\_manual}\NormalTok{(}\AttributeTok{values=}\FunctionTok{c}\NormalTok{(}\StringTok{"blue"}\NormalTok{, }\StringTok{\textquotesingle{}black\textquotesingle{}}\NormalTok{)) }\SpecialCharTok{+}
           \FunctionTok{scale\_fill\_manual}\NormalTok{(}\AttributeTok{values=}\FunctionTok{c}\NormalTok{(}\StringTok{\textquotesingle{}blue\textquotesingle{}}\NormalTok{, }\StringTok{\textquotesingle{}red\textquotesingle{}}\NormalTok{)) }\SpecialCharTok{+}
           \FunctionTok{theme\_classic}\NormalTok{() }\SpecialCharTok{+}
           \FunctionTok{xlab}\NormalTok{(}\StringTok{"logload\_t"}\NormalTok{) }\SpecialCharTok{+} \FunctionTok{ylab}\NormalTok{(}\StringTok{"logload\_tplus1"}\NormalTok{) }\SpecialCharTok{+} \FunctionTok{ggtitle}\NormalTok{(}\StringTok{" Temperature = 20C"}\NormalTok{)}
\NormalTok{p1}\SpecialCharTok{+}\NormalTok{p2}
\end{Highlighting}
\end{Shaded}

\pandocbounded{\includegraphics[keepaspectratio]{homework2_files/figure-latex/unnamed-chunk-10-1.pdf}}

\begin{Shaded}
\begin{Highlighting}[]
\NormalTok{p1}
\end{Highlighting}
\end{Shaded}

\pandocbounded{\includegraphics[keepaspectratio]{homework2_files/figure-latex/unnamed-chunk-10-2.pdf}}

For T = 12C, it appears that the modeling assumption that there is
heteroscadicity since the observed points vary greatly at lower values
of logload\_t.

\textbf{Making inference on the models}

\begin{enumerate}
\def\labelenumi{\arabic{enumi}.}
\tightlist
\item
  Using your fitted model, draw conclusions about three hypothesis
  relating to differences in Bd growth and load dynamics on frogs at
  different temperatures.
\end{enumerate}

\begin{enumerate}
\def\labelenumi{\alph{enumi}.}
\tightlist
\item
  \(log(a_{12}) = log(a_{20})\)
\item
  \(b_{12} = b_{20}\)
\item
  \(\theta_{12} = \theta_{20}\)
\end{enumerate}

When drawing your conclusions, use plots of posterior distributions and
credible intervals to support your answers.

\begin{enumerate}
\def\labelenumi{\alph{enumi}.}
\tightlist
\item
  Based on the computation ``precis(diff\_log\_a, prob=0.95)'', we can
  reject \(log(a_{12}) = log(a_{20})\).
\item
  Based on the computation ``precis(diff\_b, prob=0.95)'' we can
  conclude that it is plausible for \(b_{12} = b_{20}\).
\end{enumerate}

\begin{Shaded}
\begin{Highlighting}[]
\NormalTok{theta\_12 }\OtherTok{=}\NormalTok{ post\_12[,}\DecValTok{2}\NormalTok{]}\SpecialCharTok{/}\NormalTok{(}\DecValTok{1}\SpecialCharTok{{-}}\NormalTok{post\_12[,}\DecValTok{1}\NormalTok{])}
  
\NormalTok{theta\_20 }\OtherTok{=}\NormalTok{ post\_20[,}\DecValTok{2}\NormalTok{]}\SpecialCharTok{/}\NormalTok{(}\DecValTok{1}\SpecialCharTok{{-}}\NormalTok{post\_20[,}\DecValTok{1}\NormalTok{])}
  
\NormalTok{diff\_theta }\OtherTok{=}\NormalTok{ theta\_12 }\SpecialCharTok{{-}}\NormalTok{ theta\_20}

\FunctionTok{precis}\NormalTok{(diff\_theta, }\AttributeTok{prob=}\FloatTok{0.95}\NormalTok{)}
\end{Highlighting}
\end{Shaded}

\begin{verbatim}
##                   mean         sd       2.5%       97.5%       histogram
## diff_theta -0.07122553 0.01892606 -0.1090816 -0.03494029 ▁▁▁▁▂▃▅▇▇▅▃▁▁▁▁
\end{verbatim}

\begin{enumerate}
\def\labelenumi{\alph{enumi}.}
\setcounter{enumi}{2}
\tightlist
\item
  From the above computation, the values for diff\_theta do not overlap
  with 0, thus we can conclude that \(\theta_{12} \neq \theta_{20}\).
\end{enumerate}

\subsection{Question 3}\label{question-3}

Focus on the data from the experiment at 12 C. You may have noticed that
one of the assumptions of our model was clearly violated -- there is not
equal variance in log Bd load at time \(t + 1\) across different values
of log Bd load at time \(t\). In particular, it looks like variance in
log Bd load at time \(t + 1\) decreases as log Bd load at time \(t\) get
larger. Failing to account for these differences in variances can
substantially affect any inference we make about Bd growth dynamics.
Therefore, we want to model the patterns we see in variance.

For Bayesian analysis, modeling heterogeneity in variance is
straight-forward. Consider the generic model with predictor variable
\(x\) and response variable \(y\) and \(i = 1, \dots, n\) observations.
My Bayesian linear regression without heterogeneity in variance might
look like

\[
\begin{aligned}
y_i &\sim \text{Normal}(\mu_i, \sigma) \\
\mu_i &= \beta_0 + \beta_1 x_i \\
\beta_0 &\sim \text{Normal}(0, 3) \\
\beta_1 &\sim \text{Normal}(0, 3) \\
\sigma &\sim \text{Exponential}(1)
\end{aligned}
\]

Let's now say that the variance in \(y\) clearly changes with \(x\). I
can update my model by allowing \(\sigma\) to be a function of \(x\)
(\(f(x)\)).

\[
\begin{aligned}
y_i &\sim \text{Normal}(\mu_i, \sigma_i) \\
\mu_i &= \beta_0 + \beta_1 x_i \\
\beta_0 &\sim \text{Normal}(0, 3) \\
\beta_1 &\sim \text{Normal}(0, 3) \\
\sigma_i &= f(x_i)
\end{aligned}
\] So, you can model changes in variance, just like you model changes in
the mean. However, be careful that \(\sigma\) has a clear lower bound of
0! So choose your functions appropriately so they don't go below zero
(alternatively, you could model \(\log(\sigma)\) which would eliminate
this problem). A common choice is \(\sigma_i = s e^{\alpha_1 x_i}\).
With this function, if \(x\) is centered (i.e., a mean of 0), then \(s\)
is the standard deviation in \(y\) when \(x\) is at its mean value.
\(\alpha_1\) describes how much log \(\sigma\) changes (increases or
decreases) with an increase in \(x\).

We can then write our model as

\[
\begin{aligned}
y_i &\sim \text{Normal}(\mu_i, \sigma_i) \\
\mu_i &= \beta_0 + \beta_1 x_i \\
\beta_0 &\sim \text{Normal}(0, 3) \\
\beta_1 &\sim \text{Normal}(0, 3) \\
\sigma_i &= s e^{\alpha_1 x_i} \\
s &\sim \text{Prior} \\
\alpha_1 &\sim \text{Prior}
\end{aligned}
\] where we have one additional parameter to estimate, \(\alpha_1\). I
am purposefully not specifying the forms of the priors as this is part
of what you will have to do.

\textbf{Fitting the model}

\begin{enumerate}
\def\labelenumi{\arabic{enumi}.}
\tightlist
\item
  Fully write-out the model with non-constant variance for Bd load
  dynamics on frogs at 12 C (you don't need to do it at 20 C).
\item
  Clearly specify what prior distributions you are choosing for your new
  parameters and why.
\item
  Fit the model using \texttt{quap} and answer the question: Does our
  model support the hypothesis that variance in Bd load at time
  \(t + 1\) is decreasing with increasing Bd load at time \(t\)? Justify
  your answer with a plot, point estimate, and credible interval.
\end{enumerate}

\[
\begin{aligned}
log(x(t+1))_T &\sim \text{Normal}(log(\mu_T), \sigma) \\
log(\mu_T) &= log(a_{T}) + b_{T} [log(x(t))_T - log(\bar{x}_T)] \\
log(a_{T}) &\sim \text{Normal}(log(\bar{x}_T), 3) \\
b_{T} &\sim \text{Normal}(0, 3) \\
\sigma_i &= s e^{\alpha_1 x_i} \\
s &\sim \text{Exponential(1)} \\
\alpha_1 &\sim \text{Normal}(0, 3)
\end{aligned}
\]

Reasoning for priors:

\(s \sim \text{Exponential}(1)\): This was chosen because the support of
the Beta distribution is {[}0,1{]}, which enforces the face that we
require \(\sigma_i\) to be strictly positive as it becomes the variance
for the likelihood. Note: the probability of s is 0 is measure zero.
Having \(s \in [0,1]\) will allow us to scale the intensity of the
exponential factor in the term \(\sigma_i\).
\(\alpha_1 \sim \text{Normal(0,3)}\): This will allow the exponent to be
either positive or negative, which does not change the overall sign of
\(sigma_i\). The other priors are left the same from question 2.

\begin{Shaded}
\begin{Highlighting}[]
\CommentTok{\# fitting the model with \textquotesingle{}quap\textquotesingle{} }
\CommentTok{\#library(invgamma)}
\NormalTok{fit\_mod\_3 }\OtherTok{=} \FunctionTok{quap}\NormalTok{(}
             \FunctionTok{alist}\NormalTok{(}
\NormalTok{                logload\_tplus1 }\SpecialCharTok{\textasciitilde{}} \FunctionTok{dnorm}\NormalTok{(mu, sigma\_i),}
\NormalTok{                mu }\OtherTok{\textless{}{-}}\NormalTok{ prior\_log\_a\_12 }\SpecialCharTok{+}\NormalTok{ prior\_b}\SpecialCharTok{*}\NormalTok{scaled\_weight,}
\NormalTok{                prior\_log\_a\_12 }\SpecialCharTok{\textasciitilde{}} \FunctionTok{dnorm}\NormalTok{(mu\_12, }\DecValTok{3}\NormalTok{),}
\NormalTok{                prior\_b }\SpecialCharTok{\textasciitilde{}} \FunctionTok{dnorm}\NormalTok{(}\DecValTok{0}\NormalTok{, }\DecValTok{3}\NormalTok{),}
\NormalTok{                sigma\_i }\OtherTok{\textless{}{-}}\NormalTok{ s}\SpecialCharTok{*}\FunctionTok{exp}\NormalTok{(alpha }\SpecialCharTok{*}\NormalTok{ scaled\_weight), }
\NormalTok{                alpha }\SpecialCharTok{\textasciitilde{}} \FunctionTok{dnorm}\NormalTok{(}\DecValTok{0}\NormalTok{, }\DecValTok{3}\NormalTok{),}
\NormalTok{                s }\SpecialCharTok{\textasciitilde{}} \FunctionTok{dexp}\NormalTok{(}\FloatTok{0.5}\NormalTok{)}
\NormalTok{             ), }\AttributeTok{data =}\NormalTok{ bd\_12)}
\FunctionTok{precis}\NormalTok{(fit\_mod\_3, }\AttributeTok{prob=}\FloatTok{0.95}\NormalTok{)}
\end{Highlighting}
\end{Shaded}

\begin{verbatim}
##                       mean         sd       2.5%       97.5%
## prior_log_a_12  5.14063664 0.19929350  4.7500286  5.53124472
## prior_b         0.88599722 0.05451699  0.7791459  0.99284856
## alpha          -0.09347156 0.02475625 -0.1419929 -0.04495021
## s               1.98996065 0.12818816  1.7387165  2.24120482
\end{verbatim}

Does our model support the hypothesis that variance in Bd load at time
\(t + 1\) is decreasing with increasing Bd load at time \(t\)?\\
Justify your answer with a plot, point estimate, and credible interval.

\begin{Shaded}
\begin{Highlighting}[]
\CommentTok{\# plot}
\CommentTok{\# Generate mean predictions}

\CommentTok{\# Step 1: Draw beta0 and beta1}
\NormalTok{post\_Q3 }\OtherTok{=} \FunctionTok{extract.samples}\NormalTok{(fit\_mod\_3, }\AttributeTok{n=}\DecValTok{10000}\NormalTok{)}
\NormalTok{log\_a }\OtherTok{=}\NormalTok{ post\_Q3[, }\DecValTok{1}\NormalTok{]}
\NormalTok{b }\OtherTok{=}\NormalTok{ post\_Q3[, }\DecValTok{2}\NormalTok{]}
\NormalTok{alpha }\OtherTok{=}\NormalTok{ post\_Q3[,}\DecValTok{3}\NormalTok{]}
\NormalTok{s }\OtherTok{=}\NormalTok{ post\_Q3[, }\DecValTok{4}\NormalTok{]}


\CommentTok{\# Step 2{-}4: For a logload\_t, calculate mean}
\NormalTok{ticks }\OtherTok{=} \FunctionTok{seq}\NormalTok{(}\FunctionTok{min}\NormalTok{(bd\_12}\SpecialCharTok{$}\NormalTok{logload\_t), }\FunctionTok{max}\NormalTok{(bd\_12}\SpecialCharTok{$}\NormalTok{logload\_t), }\AttributeTok{len=}\DecValTok{50}\NormalTok{)}
\NormalTok{plogload\_t\_Q3\_scaled }\OtherTok{=}\NormalTok{ ticks }\SpecialCharTok{{-}} \FunctionTok{mean}\NormalTok{(bd\_12}\SpecialCharTok{$}\NormalTok{logload\_t)}
\NormalTok{logload\_t\_Q3\_mean }\OtherTok{=} \FunctionTok{sapply}\NormalTok{(plogload\_t\_Q3\_scaled, }\ControlFlowTok{function}\NormalTok{(x) log\_a }\SpecialCharTok{+}\NormalTok{ x}\SpecialCharTok{*}\NormalTok{b)}

\NormalTok{median\_logload\_t\_Q3 }\OtherTok{=} \FunctionTok{apply}\NormalTok{(logload\_t\_Q3\_mean, }\DecValTok{2}\NormalTok{, median)}
\NormalTok{lowerupper\_pred }\OtherTok{=} \FunctionTok{apply}\NormalTok{(logload\_t\_Q3\_mean, }\DecValTok{2}\NormalTok{, }\ControlFlowTok{function}\NormalTok{(x) }\FunctionTok{quantile}\NormalTok{(x, }\FunctionTok{c}\NormalTok{(}\FloatTok{0.025}\NormalTok{, }\FloatTok{0.975}\NormalTok{)))}
\NormalTok{logload\_t\_Q3\_df }\OtherTok{=} \FunctionTok{data.frame}\NormalTok{(}\AttributeTok{med=}\NormalTok{median\_logload\_t\_Q3, }
                     \AttributeTok{lower=}\NormalTok{lowerupper\_pred[}\DecValTok{1}\NormalTok{, ], }
                     \AttributeTok{upper=}\NormalTok{lowerupper\_pred[}\DecValTok{2}\NormalTok{, ],}
                     \AttributeTok{logload\_t=}\NormalTok{ticks)}

\NormalTok{sigma\_i }\OtherTok{=}\NormalTok{ s }\SpecialCharTok{*} \FunctionTok{exp}\NormalTok{(alpha }\SpecialCharTok{*}\NormalTok{ bd\_12}\SpecialCharTok{$}\NormalTok{scaled\_weight)}
\end{Highlighting}
\end{Shaded}

\begin{verbatim}
## Warning in alpha * bd_12$scaled_weight: longer object length is not a multiple
## of shorter object length
\end{verbatim}

\begin{Shaded}
\begin{Highlighting}[]
\NormalTok{logload\_t\_Q3\_distribution }\OtherTok{=} \FunctionTok{sapply}\NormalTok{(plogload\_t\_Q3\_scaled, }\ControlFlowTok{function}\NormalTok{(x) }\FunctionTok{rnorm}\NormalTok{(n, log\_a }\SpecialCharTok{+}\NormalTok{ x}\SpecialCharTok{*}\NormalTok{b, sigma\_i))}

\NormalTok{median\_logload\_t\_Q3\_dist }\OtherTok{=} \FunctionTok{apply}\NormalTok{(logload\_t\_Q3\_distribution, }\DecValTok{2}\NormalTok{, median)}

\NormalTok{lowerupper\_logload\_t\_Q3\_dist }\OtherTok{=} \FunctionTok{apply}\NormalTok{(logload\_t\_Q3\_distribution, }\DecValTok{2}\NormalTok{, }\ControlFlowTok{function}\NormalTok{(x) }\FunctionTok{quantile}\NormalTok{(x, }\FunctionTok{c}\NormalTok{(}\FloatTok{0.025}\NormalTok{, }\FloatTok{0.975}\NormalTok{)))}
\NormalTok{logload\_t\_Q3\_dist\_df }\OtherTok{=} \FunctionTok{data.frame}\NormalTok{(}\AttributeTok{med=}\NormalTok{median\_logload\_t\_Q3\_dist, }
                     \AttributeTok{lower=}\NormalTok{lowerupper\_logload\_t\_Q3\_dist[}\DecValTok{1}\NormalTok{, ], }
                     \AttributeTok{upper=}\NormalTok{lowerupper\_logload\_t\_Q3\_dist[}\DecValTok{2}\NormalTok{, ],}
                     \AttributeTok{logload\_t=}\NormalTok{ticks)}

\FunctionTok{ggplot}\NormalTok{() }\SpecialCharTok{+} \FunctionTok{geom\_point}\NormalTok{(}\AttributeTok{data=}\NormalTok{bd\_12, }\FunctionTok{aes}\NormalTok{(}\AttributeTok{x=}\NormalTok{logload\_t, }\AttributeTok{y=}\NormalTok{logload\_tplus1, }\AttributeTok{color=}\StringTok{"Observed"}\NormalTok{)) }\SpecialCharTok{+}
           \FunctionTok{geom\_line}\NormalTok{(}\AttributeTok{data=}\NormalTok{logload\_t\_Q3\_df, }\FunctionTok{aes}\NormalTok{(}\AttributeTok{x=}\NormalTok{logload\_t, }\AttributeTok{y=}\NormalTok{med, }\AttributeTok{color=}\StringTok{"Mean prediction"}\NormalTok{)) }\SpecialCharTok{+}
           \FunctionTok{geom\_ribbon}\NormalTok{(}\AttributeTok{data=}\NormalTok{logload\_t\_Q3\_df, }\FunctionTok{aes}\NormalTok{(}\AttributeTok{x=}\NormalTok{logload\_t, }\AttributeTok{ymin=}\NormalTok{lower, }\AttributeTok{ymax=}\NormalTok{upper, }\AttributeTok{fill=}\StringTok{"95\% CI}\SpecialCharTok{\textbackslash{}n}\StringTok{around mean"}\NormalTok{), }\AttributeTok{alpha=}\FloatTok{0.2}\NormalTok{) }\SpecialCharTok{+}
           \FunctionTok{geom\_ribbon}\NormalTok{(}\AttributeTok{data=}\NormalTok{logload\_t\_Q3\_dist\_df, }\FunctionTok{aes}\NormalTok{(}\AttributeTok{x=}\NormalTok{logload\_t, }\AttributeTok{ymin=}\NormalTok{lower, }\AttributeTok{ymax=}\NormalTok{upper, }\AttributeTok{fill=}\StringTok{"95\% CI}\SpecialCharTok{\textbackslash{}n}\StringTok{around prediction"}\NormalTok{), }\AttributeTok{alpha=}\FloatTok{0.2}\NormalTok{) }\SpecialCharTok{+}
           
       
       \FunctionTok{scale\_color\_manual}\NormalTok{(}\AttributeTok{values=}\FunctionTok{c}\NormalTok{(}\StringTok{"blue"}\NormalTok{, }\StringTok{\textquotesingle{}black\textquotesingle{}}\NormalTok{)) }\SpecialCharTok{+}
           \FunctionTok{scale\_fill\_manual}\NormalTok{(}\AttributeTok{values=}\FunctionTok{c}\NormalTok{(}\StringTok{\textquotesingle{}blue\textquotesingle{}}\NormalTok{, }\StringTok{\textquotesingle{}red\textquotesingle{}}\NormalTok{)) }\SpecialCharTok{+}
           \FunctionTok{theme\_classic}\NormalTok{() }\SpecialCharTok{+}
           \FunctionTok{xlab}\NormalTok{(}\StringTok{"logload\_t"}\NormalTok{) }\SpecialCharTok{+}
       \FunctionTok{ylab}\NormalTok{(}\StringTok{"logload\_tplus1"}\NormalTok{) }\SpecialCharTok{+}
       \FunctionTok{ggtitle}\NormalTok{(}\StringTok{"Temperature 12 C with changing variance"}\NormalTok{)}
\end{Highlighting}
\end{Shaded}

\pandocbounded{\includegraphics[keepaspectratio]{homework2_files/figure-latex/unnamed-chunk-13-1.pdf}}

\subsubsection{Bonus question: Propogating uncertainty to model
simulation}\label{bonus-question-propogating-uncertainty-to-model-simulation}

Use your model you fit in Question 3 to simulate 1000 predicted
trajectories of log Bd growth on an individual frog that are 10 time
steps long, propagating your uncertainty in your parameter estimates
through your model simulations. Assume that all trajectories start with
a log Bd load of 0.

To get started, remember that the dynamics of Bd growth on a frog are
described by

\[
\log(\mu(t + 1)) = \log(a) + b \log(x(t))
\] This is just an update equation and, given a starting value (i.e.,
log Bd load is 0), you can draw a realization of Bd load at the next
time step. For a given set of parameters, you will want to repeat this
10 times, draw a new set of parameters, and do it again. Plot the 1000
trajectories and summarize the uncertainty in the trajectories using
95\% credible ribbons (see \texttt{geom\_ribbon}).

\end{document}
